\chapter{AWS認定の受かり方}

「全冠」と聞くと、ひたすら過去問を丸暗記する姿を想像するかもしれません。しかし、私の戦略は真逆でした。いわゆる「ブレインダンプ(問題流出サイト)」は使わず、公式リソースと実践で合格ラインに乗せることを重視しました。

本章では、忙しい社会人でも再現しやすい形で、私が実際に回していた学習サイクルをまとめます。

\section{この章でできるようになること}
\begin{itemize}
	\item 試験ガイド(Exam Guide)を「学習TODO」に変換する手順が分かります
	\item 公式ブログと生成AIを使って、教材収集と理解を高速化する方法が分かります
	\item 本番当日の時間配分と、集中力を落としにくい準備が分かります
\end{itemize}

\section{試験ガイドはバイブル}
学習の第一歩は、必ずAWS公式サイトの\textbf{試験ガイド(Exam Guide)}から始めます。試験ガイドは「出題範囲」だけでなく、「その資格者に何が期待されているか(判断の軸)」が言語化されています。\footnote{AWS Certification 公式: \url{https://aws.amazon.com/jp/certification/}}

私の読み方はシンプルです。試験ガイドを一度通読したら、次の観点でマーカーを引きます。
\begin{enumerate}
	\item \textbf{知らない用語}(例: サービス名、機能名、暗号方式、ネットワーク用語)
	\item \textbf{構成が絵として浮かばない文章}(例: 「耐障害性」「運用負荷の最小化」「コスト最適化」などの非機能要件)
	\item \textbf{比較があいまいなペア}(例: DataSync と Transfer Family、Kinesis Data Streams と Firehose)
\end{enumerate}

ここでマーカーを引いた箇所が、そのまま「あなた専用の学習TODO」になります。

\section{公式リソースを主戦場にする}
学習リソースは世の中にたくさんありますが、私が軸にしたのは次の2つです。
\begin{itemize}
	\item 公式ドキュメント(一次情報)
	\item 公式ブログ(設計判断の背景が書かれていることが多い)\footnote{AWS公式ブログ: \url{https://aws.amazon.com/jp/blogs/}}
\end{itemize}

認定試験(特に上位資格)で問われやすいのは、単なる機能の暗記ではなく、「要件と制約を満たすために、どの設計を選ぶか」です。公式ブログでケーススタディを読むと、意思決定の理由が残っていることが多く、試験のシナリオ問題に直結します。

\section{生成AIは「検索」と「比較」に使う}
「公式ブログは良いが、記事が多すぎて探せない」という課題は、生成AIに肩代わりさせるのが効きます。私は主に次の2用途で使っていました。

\subsection{(1) 教材のキュレーターとして使う}
AIに「読むべき記事候補」を出させ、最後は自分で公式ソースを開いて確認します。

\begin{tcolorbox}[title=Tip: 教材探し用プロンプト]
\begin{lstlisting}
私は AWS Certified DevOps Engineer - Professional の試験対策をしています。
「セキュリティとガバナンス」に関して理解を深めたいです。
AWS公式ブログ(aws.amazon.com/jp/blogs/)の中から、図解があり学習効果が高い記事を10件提案してください。
各記事について、1) この記事を読む理由 2) 試験で問われやすい観点(要件/制約/落とし穴)を箇条書きで示してください。
最後に、提案のうち「最初に読むべき順番」を示してください。
\end{lstlisting}
\end{tcolorbox}

\subsection{(2) 概念の整理・比較に使う}
試験勉強中に詰まりやすいのは、似たサービスの使い分けです。AIには「比較表のドラフト」を作らせ、最後は公式ドキュメントで裏取りします。

\begin{itemize}
	\item ユースケース(何に向くか)
	\item 制約(何ができないか、どこで詰まりやすいか)
	\item 運用負荷(誰が何を運用するか)
	\item コストの考え方(ざっくりでよいが、断定はしない)
\end{itemize}

\section{必要なのは暗記ではなく、理解と手順化}
私は「理解したつもり」を避けるために、学習を次の3段階に分けていました。
\begin{enumerate}
	\item \textbf{言葉で説明できる}: サービスの目的と制約を説明できる
	\item \textbf{絵で説明できる}: 典型アーキテクチャを図として描ける
	\item \textbf{問題で再現できる}: シナリオ問題で、要件と制約から選択肢を絞れる
\end{enumerate}

この3段階のうち、最後まで行かないと本番で点になりにくいです。「問題が解ける形」に変換するところまでを、学習のゴールに置きます。

\section{Skill Builderで「擬似的な実務経験」を補う}
机上学習だけでは、どうしても「操作の手触り」が足りないことがあります。私は必要に応じて、AWS Skill Builderのコンテンツを使って補いました。\footnote{AWS Skill Builder: \url{https://skillbuilder.aws/}}

特に未経験者に効くのは、シナリオ型で手を動かすコンテンツです。コンソール操作やトラブルシューティングの流れが入ると、問題文の情景が浮かびやすくなります。

\section{優れたアーキテクトたれ(判断の軸を持つ)}
どの資格でも効く「判断の軸」は、突き詰めると似ています。私はWell-Architectedの考え方を、日常的なチェックリストとして使っていました。\footnote{AWS Well-Architected Framework: \url{https://docs.aws.amazon.com/wellarchitected/latest/framework/welcome.html}}

\begin{itemize}
	\item \textbf{運用負荷}: なるべくマネージドで、手作業を減らせないか
	\item \textbf{信頼性}: 単一障害点(SPOF)が残っていないか
	\item \textbf{セキュリティ}: 最小権限、境界、ログの3点が揃っているか
	\item \textbf{コスト}: いまの要件で、過剰スペックになっていないか
\end{itemize}

\section{学習モチベーションを保つための設計}
私の経験では、モチベーションは「気合」より「設計」で保つ方が再現性があります。
\begin{enumerate}
	\item 1日を\textbf{15分単位}に割り、隙間時間のタスクを決める
	\item 学習ログを残し、\textbf{やったことが可視化}される状態にする
	\item 「今日はここまで」を小さく切って、\textbf{毎日終われる}ようにする
\end{enumerate}

\section{試験当日の時間の使い方}
試験当日は、知識だけでなく時間配分と集中力が勝負になります。私がやっていたのは次のルーティンです。
\begin{enumerate}
	\item 最後の設問まで一度進み、\textbf{重い問題に印}を付ける
	\item 2周目で「印の問題」を処理する
	\item 最後に、\textbf{設問の条件(must / should)}だけを見直す
\end{enumerate}

加えて、言語サポート(いわゆるESL)による時間延長の有無は、受験環境によって異なります。受験時点の試験ポータルの案内を必ず確認してください。\footnote{Pearson VUE(受験ポータル): \url{https://www.pearsonvue.com/}}

\section{行きつけのテストセンターを作る}
最後に、軽視されがちですが環境は重要です。同じ実力でも、周囲の音や机・椅子で集中力が削られることがあります。
\begin{itemize}
	\item \textbf{静寂性}: 外部の音が気にならないか
	\item \textbf{機材}: モニターや椅子がストレスにならないか
	\item \textbf{空調}: 体温が乱れないか
\end{itemize}

「いつもの場所で、いつものように解く」状態を作れると、本番の緊張が一段下がります。

\section{まとめ}
\begin{itemize}
	\item 試験ガイドは「読む」より先に「学習TODO」に変換します
	\item 生成AIは検索と比較に使い、最後は必ず一次情報で裏取りします
	\item 本番は知識だけでなく、時間配分と環境で勝率が変わります
\end{itemize}