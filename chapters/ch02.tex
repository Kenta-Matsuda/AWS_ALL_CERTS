\chapter{AWS認定の受かり方}
「全冠」と聞くと、ひたすら過去問を丸暗記する姿を想像するかもしれません。しかし、私の戦略は真逆です。私は、いわゆる「ブレインダンプ(問題流出サイト)」を一切利用しません1。それは規約違反のリスクがあるだけでなく、何よりエンジニアとしての実力がつかないからです。
本章では、月40時間の残業を抱える私が、公式リソースと生成AIを組み合わせ、最短かつ正攻法で合格を勝ち取ってきた「実務直結型」の学習メソッドを公開します。

\section{試験ガイドはバイブル}
学習の第一歩は、必ずAWS公式サイトの**「試験ガイド(Exam Guide)」**から始まります。多くの受験者はこれをさらっと流し読みするだけですが、それはあまりに勿体ない。試験ガイドは、出題範囲を示すだけでなく、「AWSがその資格者に何を求めているか」が凝縮された聖典です2。
私の読み方はこうです。まずガイドを隅々まで読み、**「知らない単語」や「具体的なアーキテクチャ構成がイメージできない箇所」**を徹底的にハイライトします3。例えば「耐障害性の高いAIシステム」という記述があっても、具体的なサービス(Step Functionsのサーキットブレーカーパターンなど)が脳内に浮かばなければ、そこがあなたの弱点(学習すべきポイント)です。この「解像度の低い部分」を特定することこそが、試験対策のスタートラインです。

\section{公式リソースを有効活用しよう}
皆さんは学習リソースとして何を使っていますか? UdemyやBlack Beltも有用ですが、私が最も推奨するのは**「AWS公式ブログ(AWS Blogs)」**です4。
なぜなら、認定試験(特にProfessionalやSpecialtyなどの上位資格)で問われるのは、単なる機能の有無ではなく、「要件を満たすために、どのようなアーキテクチャを設計すべきか(トレードオフの判断)」だからです。公式ブログの「ケーススタディ」や「アーキテクチャ解説」は、まさにこの**「設計の意思決定プロセス」**を学ぶのに最適な教材です5。「なぜここでSQSではなくKinesisを採用したのか?」「なぜLambdaをVPC内に入れたのか?」——ブログを通して設計思想を追体験することで、試験特有のシナリオ問題に対する「判断力」が劇的に向上します。

「AWS公式ブログが良いのは分かったが、記事が多すぎて探せない」。それが最大の難点です。そこで私は、**生成AI(ChatGPTなど)**を学習の相棒としてフル活用しました6。具体的には、以下の2つの用途でAIを使役します。
① 教材のキュレーターとして使う
公式ブログの検索性の悪さは、AIにカバーさせます。私は以下のようなプロンプトを使って、自分だけの「必読記事リスト」を作成していました7。
プロンプト例:「私はAWS Certified DevOps Engineer - Professionalの試験対策をしています。ドメイン6の『セキュリティとガバナンス』、特にOrganizationsやControl Towerに関する理解を深めたいです。AWS公式ブログの中から、アーキテクチャ図が含まれたおすすめの記事を10件探し、その選定理由と共に提示してください」
これにより、膨大なアーカイブの中から、今の自分に必要な「当たり記事」だけをピンポイントで学習できます。
② 概念の整理・比較に使う
試験勉強中、「DataSyncとTransfer Familyの違いは?」「Kinesis Data StreamsとFirehoseの使い分けは?」といった疑問にぶつかることがあります。そんな時こそ、AIに**「比較表」**を作らせましょう6。「それぞれのユースケース、制約、コスト構造を表にまとめて」と指示すれば、ドキュメントを何ページも行ったり来たりする必要はありません。AIが出した答えを公式ドキュメント(一次情報)で裏取り確認する。このサイクルが、学習効率を爆発的に高めます。
\section{必要なのは暗記ではなく、理解}
アウトプットとインプットのバランスの話。
机上の学習だけでは限界がある——そう感じた時に投資すべきなのが、公式学習プラットフォーム「AWS Skill Builder」のサブスクリプションです9。月額料金(約\$29)は決して安くありませんが、ここには**「SimuLearn」**という強力なコンテンツがあります10。
SimuLearnは、架空のビジネスシナリオに沿って、実際にAWS環境を構築・トラブルシューティングするゲーム型学習です。「インフラ構築の経験がない」「実務で触ったことのないサービスが出題される」というハンデは、このSimuLearnで**「擬似的な実務経験」を買うことで解消できます。また、ここで提供される「Official Practice Question Set」**は、出所の怪しい問題集とは異なり、AWS公式が品質を保証した問題です。実力の最終確認にはこれ以上のものはありません10。

\section{優れたアーキテクトたれ}
設計のベストプラクティスを意識する。

\section{学習モチベーションを保つためには}

\section{試験時間の過ごし方}
ここで、試験当日に使える「公認の裏技」を紹介しましょう。「ESL +30」という制度をご存じでしょうか11?これは「English as a Second Language」の略で、英語が母国語でない受験者が申請すれば、試験時間が30分延長される仕組みです。
日本語で受験する場合でも申請可能です。SAP(Solutions Architect - Professional)やAIP(AI Practitioner)などの試験は、長文のシナリオを読み解く必要があり、時間は常にギリギリです。この「追加の30分」が、見直しの時間を生み、合否を分ける精神的な余裕に繋がります。申し込み前に申請ボタンを押すだけ。デメリットは一切ありません。使わない手はないでしょう。

\section{行きつけのテストセンターを作ろう}
全冠はもはや「メンタルスポーツ」です。
最後に強調したいのは、環境の重要性です。3時間、4時間にも及ぶ試験は、もはや頭脳戦というより**「メンタルスポーツ」**です12。私は過去、隣の部屋で英語のスピーキングテストが行われているテストセンターに当たり、集中力を削がれて地獄を見ました。
静寂性: 外部の音が聞こえないか。
機材: モニターの大きさや椅子の座り心地は適切か。
空調: 暑すぎたり寒すぎたりしないか。
いくつか会場を試し、自分が最もパフォーマンスを発揮できる**「行きつけのテストセンター(ホームグラウンド)」**を見つけてください。「いつもの場所で、いつものように解く」。このルーティンが、本番のプレッシャーからあなたを守ってくれます。
コードやコマンドを多用するドキュメント向け。
\begin{tcolorbox}[title=コマンド例]
\begin{lstlisting}
$ aws s3 ls
\end{lstlisting}
\end{tcolorbox}