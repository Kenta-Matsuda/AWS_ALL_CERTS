\chapter{択一式問題の解答技術}

\begin{tcolorbox}[title=この章でできるようになること, colback=blue!3, colframe=blue!60!black]
\begin{itemize}
    \item[□] 長文問題を「要件」と「制約」に分解して読む手順が分かります
    \item[□] キーワードからサービス候補を素早く連想し、選択肢を絞り込めます
    \item[□] アンチパターンで即消去し、最後の2択は設計思想で判断できます
\end{itemize}
\end{tcolorbox}

ありがとうございます。これまでの記事ソースにある「効率重視」「実務的な視点」という著者のスタンスをベースに、表紙にある「四択テクニック」「消去法」「複数回答」といったキーワードを、再現性のある技術として体系化しました。

単なる受験テクニックにとどまらず、AWSの設計思想(Well-Architected)に基づいたロジカルな解法として、第4章を構成します。
\footnote{AWS Well-Architected Framework: \url{https://docs.aws.amazon.com/wellarchitected/latest/framework/welcome.html}}

第2章、第3章では学習戦略について述べましたが、本章では「試験本番、モニターの前でどう振る舞うか」という戦術にフォーカスします。

AWS認定試験、特に上位資格(Professional / Specialty)の特徴は、\textbf{「圧倒的な文章量」}と\textbf{「紛らわしい選択肢」}です。

私が残業続きの中で短期合格できたのは、問題を一言一句精読していたからではありません。問題文から「要件」というシグナルを抽出し、AWSの設計思想に合致しないノイズ(選択肢)を瞬時に切り捨てる「フィルタリング技術」を確立していたからです。

ここでは、その技術を5つのステップで解説します。

\section{スキャニング技術:「要件」と「制約」を因数分解する}

長文のシナリオ問題(特にSAPやDOP、AIP)を小説のように頭から読んでいては、時間が足りません。問題文は読むのではなく、スキャン(走査)してください。
探すべきキーワードは、以下の2種類だけです。

\begin{enumerate}
    \item \textbf{機能要件(What)}: 何をしたいのか?
        \begin{itemize}
            \item 例: 「オンプレミスからデータベースを移行したい」「動画をトランスコードしたい」
        \end{itemize}
    \item \textbf{非機能要件・制約(How)}: 何を最優先するのか?
        \begin{itemize}
            \item 例: 「\textbf{コスト効率}を最大化する」
            \item 例: 「\textbf{運用オーバーヘッド}を最小にする(=マネージドサービスを使え)」
            \item 例: 「\textbf{ミリ秒単位}のレイテンシーが必要(=キャッシュかDynamoDB)」
            \item 例: 「\textbf{コードの変更なし}で移行する(=リホスト)」
        \end{itemize}
\end{enumerate}

問題文の最後の一文(「最もコスト効率が良いソリューションはどれか?」など)を最初に確認し、その「色眼鏡」をかけてから本文をスキャンすると、正解の候補が自動的に絞り込まれます。

\section{「四択テクニック」:キーワード・リンク法}

表紙にある「四択テクニック」の正体は、\textbf{「要件とサービスの対応」を即時に変換(マッピング)する}ことです。
AWS認定には、特定のキーワードが出たら、ほぼ自動的に正解となるサービスの組み合わせ(鉄板パターン)が存在します。これを知っているだけで、思考時間を数秒に短縮できます。

\begin{center}
\begin{tabular}{p{0.42\linewidth} p{0.52\linewidth}}
\hline
\textbf{キーワード(要件)} & \textbf{反射的に想起すべきサービス・機能} \\
\hline
\textbf{「ミリ秒単位」「キーバリュー」} & \textbf{DynamoDB} \\
\textbf{「標準SQL」「データウェアハウス」} & \textbf{Amazon Redshift} \\
\textbf{「HPC」「並列分散処理」「Lustre」} & \textbf{Amazon FSx for Lustre} \\
\textbf{「運用負荷の軽減」「サーバー管理不要」} & \textbf{AWS Lambda / AWS Fargate(サーバーレス)} \\
\textbf{「非構造化データ」「安価なストレージ」} & \textbf{Amazon S3} \\
\textbf{「ハイブリッド接続」「専用線」} & \textbf{AWS Direct Connect} \\
\textbf{「プロトコル変換」「IoT」} & \textbf{AWS IoT Core} \\
\hline
\end{tabular}
\end{center}

問題文に「HPC」とあり、選択肢に「EFS」と「FSx for Lustre」があれば、迷わず後者をキープします。これがキーワード・リンク法です。

\section{「消去法」の極意:アンチパターンによる即時除外}

正解を探すより、「明らかに間違い」を探す方が簡単で確実です。AWSには「やってはいけないこと(アンチパターン)」があり、これを含む選択肢は100%誤りです。

\textbf{【即時消去すべきNGワード・構成】}

\begin{enumerate}
    \item \textbf{「EC2に〇〇をインストールして...」}
        \begin{itemize}
            \item AWSの正解は基本的に「マネージドサービスの活用」です。既存のマネージドサービス(例: RDS、Amazon MQ、Amazon SES)があるのに、わざわざEC2で自前構築する選択肢は不正解寄りになりやすいです。
            \item 例外として、要件に「OSレベルのカスタマイズが必要」など明確な理由がある場合があります。
        \end{itemize}
    \item \textbf{ハードコーディング / IAMユーザーの共有}
        \begin{itemize}
            \item 「認証情報をコードに埋め込む」「IAMユーザーをチームで使い回す」といった記述は、セキュリティ的にNGなので即消去します。
        \end{itemize}
    \item \textbf{アンチパターンの組み合わせ}
        \begin{itemize}
            \item 「S3をデータベースとして使用し、トランザクション処理を行う」(S3はオブジェクトストレージであり、DBではありません)
            \item 「パブリックサブネットにデータベースを配置する」(DBはプライベート配置が基本です)
        \end{itemize}
\end{enumerate}

これらを見つけたら、選択肢の内容を検討するまでもなく、斜線を引いてください。

\section{高難易度の「複数回答」攻略アプローチ}

「3つ選択してください」といった複数回答問題は、受験者を苦しめる鬼門です。しかし、ここにも攻略法があります。それは「レイヤー分け」です。

多くの場合、正解の3つは、異なるレイヤー(層)の解決策の組み合わせになっています。
例えば、「Webアプリのパフォーマンス改善」で3つ選ぶ場合:

\begin{itemize}
    \item \textbf{レイヤー1(エッジ)}: CloudFrontを導入する
    \item \textbf{レイヤー2(コンピューティング)}: Auto Scalingを設定する
    \item \textbf{レイヤー3(データベース)}: ElastiCacheを導入する
\end{itemize}

このように、システム全体を最適化するために「互いに矛盾しない、異なる役割の選択肢」をピックアップするのがコツです。
逆に、同じレイヤーの選択肢(例:「DynamoDBを使う」と「RDSを使う」)は排他関係にあることが多く、同時に選ぶことは稀です。

\section{最後の2択で迷った時の「AWS Way」}

テクニックを駆使しても、どうしても2つの選択肢で迷うことがあります。
どちらも技術的には可能(動く構成)である場合です。その時は、「どちらがよりAWSらしいか(AWS Way)」で判断します。

\begin{itemize}
    \item \textbf{疎結合か?}: SQSやSNSを挟んでコンポーネントを分離している方が「AWSらしい」ことが多いです
    \item \textbf{ステートレスか?}: サーバーに状態を持たせず、スケーリングしやすい方が有利になりやすいです
    \item \textbf{マネージドか?}: 運用負荷をAWSにオフロードしている方が好まれやすいです
\end{itemize}

迷ったら、「自分が楽をするため(運用負荷軽減)」ではなく、「システムが自律的に動くため」の選択肢を選んでください。それがAWSの求める「アーキテクト」の姿です。

\section{まとめ}
\begin{itemize}
    \item 長文問題はスキャンし、機能要件(What)と制約(How)に分解します
    \item キーワードで候補を連想し、アンチパターンで即消去して時間を稼ぎます
    \item 最後の2択はWell-Architectedの設計思想(疎結合・ステートレス・マネージド)で判断します
\end{itemize}