\chapter{択一式問題の解答技術}

ありがとうございます。これまでの記事ソースにある「効率重視」「実務的な視点」という著者のスタンスをベースに、表紙にある**「四択テクニック」「消去法」「複数回答」**といったキーワードを、再現性のある技術として体系化しました。

単なる受験テクニックにとどまらず、AWSの設計思想(Well-Architected)に基づいたロジカルな解法として、第4章を構成します。

---

# 第4章 択一式問題の解答技術

第2章、第3章では学習戦略について述べましたが、本章では**「試験本番、モニターの前でどう振る舞うか」**という戦術にフォーカスします。
AWS認定試験、特に上位資格(Professional / Specialty)の特徴は、**「圧倒的な文章量」**と**「紛らわしい選択肢」**です。

私が残業続きの中で短期合格できたのは、問題を一言一句精読していたからではありません。問題文から「要件」というシグナルを抽出し、AWSの設計思想に合致しないノイズ(選択肢)を瞬時に切り捨てる**「フィルタリング技術」**を確立していたからです。
ここでは、その技術を5つのステップで解説します。

## 4.1 スキャニング技術:「要件」と「制約」を因数分解する

長文のシナリオ問題(特にSAPやDOP、AIP)を小説のように頭から読んでいては、時間が足りません。問題文は読むのではなく、**スキャン(走査)**してください。
探すべきキーワードは、以下の2種類だけです。

1.  **機能要件(What):何をしたいのか?**
    *   例:「オンプレミスからデータベースを移行したい」「動画をトランスコードしたい」
2.  **非機能要件・制約(How):何を最優先するのか?**
    *   例:「**コスト効率**を最大化する」
    *   例:「**運用オーバーヘッド**を最小にする(=マネージドサービスを使え)」
    *   例:「**ミリ秒単位**のレイテンシーが必要(=キャッシュかDynamoDB)」
    *   例:「**コードの変更なし**で移行する(=リホスト)」

問題文の最後の一文(「最もコスト効率が良いソリューションはどれか?」など)を最初に確認し、その「色眼鏡」をかけてから本文をスキャンすると、正解の候補が自動的に絞り込まれます。

## 4.2 「四択テクニック」:キーワード・リンク法

表紙にある「四択テクニック」の正体は、**「要件⇔サービス」の即時変換(マッピング)**です。
AWS認定には、特定のキーワードが出たら、ほぼ自動的に正解となるサービスの組み合わせ(鉄板パターン)が存在します。これを知っているだけで、思考時間を数秒に短縮できます。

| キーワード(要件) | 反射的に想起すべきサービス・機能 |
| :--- | :--- |
| **「ミリ秒単位」「キーバリュー」** | **DynamoDB** |
| **「標準SQL」「PB級のデータウェアハウス」** | **Redshift** |
| **「HPC」「並列分散処理」「Lustre」** | **FSx for Lustre** |
| **「運用負荷の軽減」「サーバー管理不要」** | **Lambda / Fargate (サーバーレス)** |
| **「非構造化データ」「安価なストレージ」** | **S3** |
| **「ハイブリッド接続」「専用線」** | **Direct Connect** |
| **「プロトコル変換」「IoT」** | **IoT Core** |

問題文に「HPC」とあり、選択肢に「EFS」と「FSx for Lustre」があれば、迷わず後者をキープします。これが**キーワード・リンク法**です。

## 4.3 「消去法」の極意:アンチパターンによる即時除外

正解を探すより、**「明らかに間違い」**を探す方が簡単で確実です。AWSには「やってはいけないこと(アンチパターン)」があり、これを含む選択肢は100%誤りです。

**【即時消去すべきNGワード・構成】**

1.  **「EC2に〇〇をインストールして...」**
    *   AWSの正解は基本的に「マネージドサービスの活用」です。既存のマネージドサービス(例:RDS, MQ, SES)があるのに、わざわざEC2で自前構築する選択肢は、9割方「不正解」です。
    *   ※例外:要件に「OSレベルのカスタマイズが必要」とある場合のみ。
2.  **ハードコーディング / IAMユーザーの共有**
    *   「認証情報をコードに埋め込む」「IAMユーザーをチームで使い回す」といった記述は、セキュリティ的に絶対NGなので即消去。
3.  **アンチパターンの組み合わせ**
    *   「S3をデータベースとして使用し、トランザクション処理を行う」(S3はオブジェクトストレージであり、DBではない)
    *   「パブリックサブネットにデータベースを配置する」(DBはプライベートに置くのが鉄則)

これらを見つけたら、選択肢の内容を検討するまでもなく、斜線を引いてください。

## 4.4 高難易度の「複数回答」攻略アプローチ

「3つ選択してください」といった複数回答問題は、受験者を苦しめる鬼門です。しかし、ここにも攻略法があります。それは**「レイヤー分け」**です。

多くの場合、正解の3つは、異なるレイヤー(層)の解決策の組み合わせになっています。
例えば、「Webアプリのパフォーマンス改善」で3つ選ぶ場合:

*   **レイヤー1(エッジ)**: CloudFrontを導入する
*   **レイヤー2(コンピューティング)**: Auto Scalingを設定する
*   **レイヤー3(データベース)**: ElastiCacheを導入する

このように、システム全体を最適化するために**「互いに矛盾しない、異なる役割の選択肢」**をピックアップするのがコツです。
逆に、同じレイヤーの選択肢(例:「DynamoDBを使う」と「RDSを使う」)は排他関係にあることが多く、同時に選ぶことは稀です。

## 4.5 最後の2択で迷った時の「AWS Way」

テクニックを駆使しても、どうしても2つの選択肢で迷うことがあります。
どちらも技術的には可能(動く構成)である場合です。その時は、**「どちらがよりAWSらしいか(AWS Way)」**で判断します。

*   **疎結合か?**: SQSやSNSを挟んでコンポーネントを分離している方が偉い。
*   **ステートレスか?**: サーバーに状態を持たせず、スケーリングしやすい方が偉い。
*   **マネージドか?**: 運用負荷をAWSにオフロードしている方が偉い。

迷ったら、「自分が楽をするため(運用負荷軽減)」ではなく、**「システムが自律的に動くため」**の選択肢を選んでください。それがAWSの求める「アーキテクト」の姿です。