\chapter{バックグラウンド別の全冠ルート}

全冠への道は一本道ではありません。ただ、確実に言えるのは、\textbf{試験同士には知識の重なりがある}ということです。ある試験で学んだ内容が、別の試験でそのまま効く場面がよくあります。

私はこの「重なり」を意識して順番を組み、忙しい時期でも学習の重複を減らして進めました。

\section{この章でできるようになること}
\begin{itemize}
	\item 自分のバックグラウンドに合わせた「取りやすい順番」を設計できます
	\item 各ルートの狙い(何が得意になり、何を補う必要があるか)が分かります
	\item 次の章・次の教材に繋がる「学習の当たり所」を把握できます
\end{itemize}

\section{全員共通: 外堀を埋めてから本丸を攻める}
ルート分岐に入る前に、全員に共通する戦略を書いておきます。

私は、いきなりSAP(Solutions Architect - Professional)に突撃するのではなく、\textbf{周辺資格で外堀を埋めてから本丸を攻める}やり方をおすすめします。私はSAPを受ける時点で、Associate 3冠やDOP、MLSなど複数資格を先に取得していました。\footnote{AWS Certification(資格一覧): \url{https://aws.amazon.com/jp/certification/}}

その状態でSAPに入ると、IAM、VPC、デプロイ戦略などは「復習」に近い感覚になります。結果として、SAP固有の論点(Organizationsやデータ転送の選択肢など)に時間を集中できました。

\section{未経験・初学者から全冠を目指す場合}
クラウドの実務経験がない方は、AWSの共通言語を獲得するところから始めるのが安全です。

\subsection{推奨ルート}
\begin{enumerate}
	\item Cloud Practitioner(CLF)
	\item Solutions Architect - Associate(SAA)
	\item Developer - Associate(DVA)
	\item SysOps Administrator - Associate(SOA)
\end{enumerate}

\subsection{狙いとコツ}
\begin{itemize}
	\item SAAは最重要拠点です。ここで「要件→設計→トレードオフ」の型を作ります
	\item DVAとSOAはSAAと範囲が被りやすいので、記憶が新しいうちに連続で取りに行きます
\end{itemize}

未経験者が詰まりやすい最大の壁は「実際の画面や操作がイメージできないこと」です。ここで暗記に寄せすぎないために、私はAWS Skill Builderのハンズオン系コンテンツを活用しました。\footnote{AWS Skill Builder: \url{https://skillbuilder.aws/}}

\section{インフラエンジニア向け 全冠ルート}
普段オンプレミスやサーバー構築に携わっている方は、ネットワークと運用の勘所を武器にできます。

\subsection{推奨ルート(例)}
\begin{enumerate}
	\item Solutions Architect - Associate(SAA)
	\item SysOps Administrator - Associate(SOA)
	\item Advanced Networking - Specialty(ANS)
	\item Security - Specialty(SCS)
	\item DevOps Engineer - Professional(DOP)
	\item Solutions Architect - Professional(SAP)
\end{enumerate}

\subsection{狙いとコツ}
\begin{itemize}
	\item SOAは監視・運用・トラブルシュートが入るため、インフラ経験と噛み合います
	\item ANSとSCSは「守り」の知識がリンクしやすく、連続で学ぶと理解が進みます
	\item DOPは運用自動化(CI/CD、デプロイ戦略)の視点を足せるため、SAP前の良い土台になります
\end{itemize}

\section{アプリケーションエンジニア向け 全冠ルート}
普段コードを書く方は、インフラの細かい設定よりも「サービスをどう組み合わせるか(疎結合)」の理解が速いことが多いです。私はこのルートの派生形で進めました。

\subsection{推奨ルート(例)}
\begin{enumerate}
	\item Developer - Associate(DVA)
	\item Solutions Architect - Associate(SAA)
	\item DevOps Engineer - Professional(DOP)
	\item Data Engineer - Associate(DEA)
\end{enumerate}

\subsection{狙いとコツ}
\begin{itemize}
	\item DVAで点としてサービスを掴み、SAAで線(アーキテクチャ)に繋げます
	\item DOPはCI/CDやデプロイ戦略が中心で、アプリ経験の延長で取り組みやすいことがあります
	\item DEAはデータ処理系(Kinesis、Glue、S3設計など)を固めるのに役立ちます
\end{itemize}

\section{セキュリティ/運用志向向け 全冠ルート}
「監査」「インシデント対応」「ガバナンス」「ログ設計」などに関心がある方は、守りの軸を先に太くすると学習が楽になります。

\subsection{推奨ルート(例)}
\begin{enumerate}
	\item Solutions Architect - Associate(SAA)
	\item Security - Specialty(SCS)
	\item SysOps Administrator - Associate(SOA)
	\item DevOps Engineer - Professional(DOP)
	\item Solutions Architect - Professional(SAP)
\end{enumerate}

\subsection{狙いとコツ}
\begin{itemize}
	\item SCSは「アイデンティティ」「データ保護」「検知と対応」の軸を作れます
	\item 運用(SOA)を組み合わせると、ログ・監視・変更管理の「実務の匂い」が乗ります
	\item SAPはガバナンス(Organizations等)や複合設計が問われやすいので、守り+運用の土台があると戦いやすいです
\end{itemize}

\section{AIエンジニア向け 全冠ルート}
データとAIは個別に取るより、\textbf{データの流れ}としてセットで攻略すると理解がつながりやすいです。

\subsection{推奨ルート(例)}
\begin{enumerate}
	\item Data Engineer - Associate(DEA)
	\item Machine Learning - Specialty(MLS)
	\item AWS Certified AI Practitioner(AIP)
\end{enumerate}

\subsection{狙いとコツ}
\begin{itemize}
	\item AIの燃料はデータです。DEAで収集・蓄積・加工(S3、Kinesis、Glue等)を固めます
	\item MLSでモデル構築の流れ(特にSageMaker周辺)を押さえると、AI領域の解像度が上がります
	\item AIPは生成AIの基礎概念とAWS上の選択肢を整理するのに向きます\footnote{AWS Certified AI Practitioner: \url{https://aws.amazon.com/jp/certification/certified-ai-practitioner/}}
\end{itemize}

\section{まとめ}
\begin{itemize}
	\item 全冠は「順番」で難易度が変わります。知識が重なる順に積み上げます
	\item SAPは最初に突撃せず、周辺資格で土台を作ってから攻めるのが再現性が高いです
	\item 未経験者はハンズオンで操作の解像度を上げ、暗記に寄りすぎないようにします
\end{itemize}