\chapter{バックグラウンド別の全冠ルート}

全冠への道は、一本道ではありません。しかし、確実に言えるのは**「試験同士には強力な相関関係(シナジー)がある」**ということです。ある試験で学んだ知識は、別の試験の40%〜60%をカバーしていることがよくあります。この「知識の重複」を戦略的に突くことで、私は月40時間の残業をしながらでも、最短ルートで全冠を達成しました。
本章では、あなたのバックグラウンドに合わせ、最も学習効率が良い(=試験勉強の重複が多い)ルートを提唱します。

ルート分岐に入る前に、全冠を目指す全員に共通する戦略をお伝えします。それは、いきなり最難関の SAP (Solutions Architect - Professional) を目指すのではなく、**「周辺資格で外堀を埋めてから本丸を攻める」**というアプローチです。
私はSAP受験時、すでにAssociate 3冠とDOP、MLSなど計9資格を持っていました1。これにより、SAP試験範囲の大部分(IAM、VPC、デプロイ戦略など)を既に理解しており、SAP固有の勉強は「AWS Organizations」や「データ転送」などに絞ることができました1。この「急がば回れ」こそが、結果的に最短で合格する秘訣です。


\section{未経験・初学者から全冠を目指す場合}
〜王道の「Associate 3兄弟」制覇ルート〜
クラウドの実務経験がない方は、AWSの共通言語を学ぶこのルートが鉄則です。
Cloud Practitioner (CLF)
まずはAWSの用語と「広くて浅い」全体像を把握します。
Solutions Architect - Associate (SAA)
ここが最重要拠点です。 全資格の基礎となる「設計の考え方」を学びます。SAAをしっかり理解していないと、後の資格ですべて苦労します。
Developer - Associate (DVA) & SysOps Administrator - Associate (SOA)
SAA取得後は、残りのAssociate 2種を連続して取ります。
ポイント: DVAとSOAは、SAAと範囲がかなり被っています。「鉄は熱いうちに打て」の精神で、記憶が新鮮なうちに一気に取り切るのが最もコスパが良いです。
★ここで差がつく!未経験者最大の壁は「実際の画面や操作がイメージできないこと」です。ここで無理に暗記に走らず、**AWS Skill Builderの「SimuLearn」**を活用してください2。ゲーム感覚で構築フローを疑似体験することで、未経験のハンデを「経験」として埋めることができます。


\section{インフラエンジニア向け 全冠ルート}
〜ネットワークと運用の強みを活かす「堅牢化」ルート〜
普段、オンプレミスやサーバー構築に携わっている方は、ネットワークやセキュリティの知識を武器にします。
SAA & SOA (SysOps)
運用監視やトラブルシューティングを含むSOAは、インフラエンジニアにとって最も親しみやすいはずです。
Advanced Networking - Specialty (ANS)
多くの人が苦手とする「ネットワーク」ですが、インフラ勢にはアドバンテージがあります。DX (Direct Connect) や VPN、Transit Gatewayなど、実務に近い知識で攻略できます。
Security - Specialty (SCS)
ANSとSCSは「守り」の要として知識がリンクします。
DevOps Engineer - Professional (DOP)
運用(SOA)と開発(DVA)のハイブリッドですが、CI/CDパイプラインやデプロイ戦略はインフラの自動化と密接に関わります。
Solutions Architect - Professional (SAP)
ここまで揃えば、SAPは「今までの総復習」に近い感覚で解けるはずです。


\section{アプリケーションエンジニア向け 全冠ルート}
〜コードとAPIを武器にする「開発者」ルート〜(※著者はこのルートの派生形です)
普段コードを書いているエンジニアは、インフラの細かい設定よりも「サービスをどう組み合わせるか(疎結合)」の理解が早いです。
Developer - Associate (DVA)
Lambda、DynamoDB、API Gatewayなど、馴染みのあるサーバーレス構成から入ります。
Solutions Architect - Associate (SAA)
DVAで点としてのサービスを知ってから、SAAで線(アーキテクチャ)として繋げます。
DevOps Engineer - Professional (DOP)
ここが戦略の肝です。 アプリエンジニアにとって、Codeシリーズ(CodeBuild/CodeDeployなど)が出題されるDOPは、SAPよりも取り組みやすいProfessional資格です。DVAの上位互換としてDOPを早期に攻略します3。
Data Engineer - Associate (DEA)
アプリ開発とデータ処理は切っても切れません。KinesisやGlueなどのデータ基盤を学びます。

\section{セキュリティ/運用志向向け 全冠ルート}

\section{AIエンジニア向け 全冠ルート}
〜最新トレンド「データ&AI」のシナジーを最大化する〜
現在、AWS認定の中で最も熱いのがAI・データ領域です。これらは個別に取るのではなく、**「データの流れ」**としてセットで攻略します。
Data Engineer - Associate (DEA)
AIの燃料は「データ」です。まずはデータの収集・蓄積(S3, Kinesis, Glue)を固めます。
Machine Learning - Specialty (MLS)
DEAで整備したデータを使って、SageMakerでモデルを構築する流れを学びます。DEAとの親和性は抜群です。
AWS Certified AI Practitioner (AIP) & Machine Learning Engineer - Associate (MLA - 予定)
最新のAIPは、生成AI(Bedrock/Q)に特化しています4。MLSの知識(プロンプトエンジニアリングやモデル評価)があれば、AIPはスムーズに合格可能です。また、2025年以降の新資格MLAへの足掛かりにもなります。