\chapter{全冠の先を見据えて}
長い旅路を経て、ついに「All Certifications(全冠)」を達成したとき、あなたの目の前にはどんな景色が広がっているでしょうか。
そして、その資格を武器に、これからエンジニアとしてどう歩んでいくべきか。
最終章では、全冠ホルダーとなった私が今感じている「リアル」と、これからのキャリア戦略についてお話しします。

\section{全冠はゴールではなくスタートラインである}

「全冠したら、もう勉強しなくていい」。そう思っていた時期が私にもありました。しかし現実は違います。AWSの世界は秒進分歩で進化しており、全冠はその変化のスタートラインに立ったことに過ぎません。

象徴的なのが、2024年に新設された**「AWS Certified AI Practitioner (AIP)」**や**「Machine Learning Engineer - Associate (MLA)」**の登場です。
私はAIPのベータ試験が開始された当日の朝一番に受験し、合格しました。なぜなら、全冠ホルダーとは「過去の知識を持っている人」ではなく、「新しい技術キャッチアップを誰よりも早く行える人」であるべきだと思うからです。

全冠の真の価値は、バッジのコレクションそのものではなく、**「AWSの全サービスがつながった状態で地図が見えている」**という状態にあります。
新しいサービスが登場したとき、「ああ、これは既存の〇〇と××を組み合わせたものだな」と瞬時に理解できる。この「高速学習サイクル」に入れたことこそが、全冠の最大の報酬です。

\section{AWS認定を実務・キャリアにどうつなげるか}

第1章でお話しした通り、私は現在、AWSとは直接関係のない業務を行うSEです。「実務で使わないなら、資格なんて意味がないのでは?」という意見もあるでしょう。
しかし、私はそうは思いません。資格学習で得た「設計思想(アーキテクチャ)」は、あらゆるITシステムに通用するからです。

### ① 「選ばれる人材」になるための準備
AWSパートナー(APN)企業にいる以上、チャンスは突然巡ってきます。その時、「やる気はあります」と言う人と、「全冠持っています。SimuLearnで擬似構築も経験済みです」と言う人、どちらに案件を任せたいかは明白です。
資格は、実務経験がない私たちが切れる**「最強のカード」**であり、チャンスを掴むための準備です。

### ② 「SimuLearn」で経験の壁を突破する
「ペーパー資格」と揶揄されるのが怖いなら、第2章で紹介した**AWS Skill Builderの「SimuLearn」**を徹底的にやり込んでください。
架空のシナリオとはいえ、実際にコンソールを操作し、トラブルシューティングを行った経験は、面接や現場での「言葉の重み」を変えます。私はサブスクリプションに課金し、これをやり込むことで「実務未経験」のコンプレックスを払拭しました。

### ③ ブレインダンプを使わなかった「自信」
もしあなたが、違法な問題集サイト(ブレインダンプ)を使って合格していたら、現場でトラブルが起きた時に「自分の知識は偽物だ」と怯えることになるでしょう。
しかし、本書のメソッド通りに公式ドキュメントと公式ブログを読み込み、悩み抜いて合格したあなたには、**「AWSの思想」**が血肉として宿っています。その自信こそが、キャリアを支える屋台骨になります。

\section{これから全冠を目指す人へ}


最後に、これからこの長い山を登ろうとしているあなたへ。

「非情報系出身」「実務未経験」「月40時間の残業」。
これらはすべて、私が全冠を目指し始めた時のスペックです。決して恵まれた環境ではありませんでした。それでも、正しい戦略(試験ガイドの分解、ブログ×AI活用、重複領域の効率化)を用いれば、短期間で頂上に立つことは可能です。

全冠は、一部の天才だけの特権ではありません。
**「正しい努力を、戦略的に継続できる」**。その証明ができるエンジニアを、市場が放っておくはずがありません。

さあ、まずは試験ガイドをダウンロードするところから始めましょう。
テストセンターの予約ボタンを押した瞬間、あなたの「全冠エンジニア」へのキャリアは始まっています。