\chapter{はじめに}

はじめまして。松田です。
この本を手に取ってくださり、ありがとうございます。

本書は、AWS認定の全冠に興味がある方に向けて、「忙しい社会人でも再現できる攻略手順」をまとめたものです。

私は富士通Japan株式会社に在籍し、これまで文教向けのマルチテナントSaaS開発に携わってきました。
AWSは専門職として長年触ってきたわけではありませんが、業務でECS on Fargate、EC2、EBSなどを触るところから学習を積み上げてきました。

また、AWS認定は\textbf{12冠(Machine Learning - Specialty(MLS)を除く12資格)}を取得し、\textbf{Japan All AWS Certifications Engineers} に申請中です。
\footnote{Japan All AWS Certifications Engineers(参考): \url{https://aws.amazon.com/jp/blogs/psa/2026-japan-all-aws-certifications-engineers-criteria/}}

\begin{tcolorbox}[title=この章でできるようになること, colback=blue!3, colframe=blue!60!black]
\begin{itemize}
	\item[□] 本書の目的と読み方が分かります
	\item[□] AWS認定・全冠の全体像と、狙う価値が分かります
	\item[□] 忙しくても前に進める学習の型(メソッド)の前提が分かります
\end{itemize}
\end{tcolorbox}

\section{自己紹介}
「AWSの全冠(全認定取得)」と聞くと、皆さんはどのような人物を想像するでしょうか?
クラウドの黎明期から業界を牽引してきたベテランエンジニア、あるいは寝食を忘れて技術書を読み漁る一部の天才たち……。もしあなたがそう思っているなら、まずはその誤解を解くところから始めさせてください。

先に、私の立ち位置を箇条書きでまとめます。
(「自分と近いかも」と思える要素があれば、この本はかなり刺さるはずです。)

\begin{itemize}
	\item 社会人3年目、25歳です
	\item 大学は非情報系で、入社時点のインフラ知識はほぼゼロでした
	\item 富士通Japan株式会社に在籍し、文教向けのマルチテナントSaaS開発に携わってきました
	\item 平均月40時間以上の残業があり、まとまった勉強時間を取りにくい時期もありました
\end{itemize}

筆者である私は現在、社会人3年目の25歳です。大学時代は非情報系の学部に所属しており、プログラミングといえばJavaとPythonを少し独学で触った程度。ネットワークやサーバーといったインフラの知識は、入社するまで完全にゼロでした。

「クラウドに強い人が全冠を取る」のではなく、\textbf{「クラウドに強くないところから、戦略で積み上げる」}。
私は、そのタイプの受験者だと思っています。

社会人1年目に配属された最初の部署でAWSと出会いました。
ただ、その部署ではAWSを実務で触る機会がほとんどありませんでした。

その後、部署異動を経て、社会人1年目の冬(2025年1月)頃から、業務でECS on Fargate、EC2、EBSなどを触るようになりました。
このタイミングで最初の資格「Cloud Practitioner (CLF)」を取得し、そこからAWS認定の勉強が本格的に始まりました。

そんな私が、平均月40時間以上の残業がある中でも、約1ヶ月に1個のペースで試験を突破してきました。

現時点では、\textbf{Machine Learning - Specialty(MLS)を除く12資格}を取得しています。
本書では、その過程で身につけた「再現性のある攻略手順」を、読者が辿れる形に分解して紹介します。

\subsection{ざっくり年表(どんな順番で進んだか)}
私の経験を「再現できる形」にするため、ここでは大雑把な流れを示します。
細かい試験順や戦略は、後続の章で具体的に解説します。

\begin{itemize}
	\item 社会人1年目: AWSと出会う(ただし最初の部署では触る機会が少ない)
	\item 2025年1月頃: 業務でECS on Fargate、EC2、EBSなどに触れ始める → CLF取得
	\item 4つ目の資格(AWS MLA)取得後: 「全冠」を明確な目標として走り始める
	\item 社会人3年目: 残業がある中で、約1ヶ月に1個ペースで受験を継続 → 全冠達成
\end{itemize}

\subsection{全冠を目指し始めたきっかけ}
私が「全冠を狙おう」と決めたのは、4つ目の資格である\textbf{AWS MLA}を取得した後でした。
せっかく4つ取ったのだから、どうせなら全部取り切ってみたい。
そして、同僚にも「自分も頑張らなきゃ」と思ってもらえるような刺激になりたい。
そう考えたのがスタートです。

さらに、\textbf{Japan All AWS Certifications Engineers} という制度を知ったことで、全冠達成が自分の中でより明確な目標になりました。
\footnote{Japan All AWS Certifications Engineers(参考): \url{https://aws.amazon.com/jp/blogs/psa/2026-japan-all-aws-certifications-engineers-criteria/}}

個人的には、ここからが「趣味の勉強」ではなく、\textbf{やり切るためのプロジェクト}になりました。

\subsection{勉強時間の作り方(忙しい前提で回す)}
平日は仕事があるので、勉強は1時間程度です。
休日は土日合わせて半日程度を目安にしていました。

基本はテレワークですが、出勤する日は移動時間も勉強に充てました。
また、日常の中で「頭を使っていない時間」を拾うことを意識していました。
たとえば、お皿を洗う間はBlack Beltを見る、といった具合です。

このルーティンができるようになってから、だいたいどの難易度の資格も1ヶ月スパンで合格していけました。

\subsection{モチベーション維持のリアル(落ちる時期はあります)}
私は幸い、一度も不合格にはならなかったのですが、モチベーション維持には苦労しました。
特に、SOA〜DEA、SAP、SCSあたりは、目標を見失ってモチベーションが低下しやすかったです。

そういうときは、あえてAWSを勉強しない時間を作りました。
趣味を楽しんだり、他のことを勉強したりしていると、またAWSのやる気が戻ってきます。
私は気分屋なので、この「気分に任せる」やり方が合っていました。

\subsection{受験順(参考)}
参考までに、私が受けた順番は次の通りです(略称も併記します)。

\begin{itemize}
	\item AWS Certified Cloud Practitioner(CLF)
	\item AWS Certified Solutions Architect - Associate(SAA)
	\item AWS Certified AI Practitioner(AIF-C01)
	\item AWS Certified Machine Learning Engineer - Associate(MLA)
	\item AWS Certified Developer - Associate(DVA)
	\item AWS Certified CloudOps Engineer - Associate(旧称: AWS Certified SysOps Administrator - Associate)(SOA)
	\item AWS Certified Data Engineer - Associate(DEA)
	\item AWS Certified DevOps Engineer - Professional(DOP)
	\item AWS Certified Solutions Architect - Professional(SAP)
	\item AWS Certified Generative AI Developer - Professional(ベータ試験)
	\item AWS Certified Security - Specialty(SCS)
	\item AWS Certified Advanced Networking - Specialty(ANS)
\end{itemize}

この順番にも意図があります。
本書では、順番をそのまま真似するのではなく、あなたの状況に合わせて最適化できるように分解して解説します。

\subsection{AI Practitionerを「最速」で受けた理由}
私は\textbf{AWS Certified AI Practitioner}のベータ試験が開始された当日の朝一番に受験し、合格しました。
なぜなら、全冠ホルダーとは「過去の知識を持っている人」ではなく、「新しい技術キャッチアップを誰よりも早く行える人」であるべきだと思うからです。

\subsection{本書に書きたいのは「根性論」ではなく「再現性」です}
私は、自分自身を「まだまだ未熟者」だと思っています。

だからこそ断言できます。全冠達成に特別な才能は不要です。必要なのは、正しい戦略と、少しの工夫だけなのです。

本書は、私が実際に積み上げてきた手順を、読者がそのまま辿れる形に分解してまとめたものです。
「忙しい」「実務が薄い」「情報が多すぎて溺れそう」という状況でも、前に進めるように設計しています。

なお、学習においては、正攻法(公式情報を軸に理解を積み上げること)を大切にします。
これは倫理の話だけでなく、合格後に現場で自信を持って説明できる状態を作るためでもあります。

私は、なるべく腹落ちして理解したいタイプです。
そのため、分からないところはAIに徹底的に質問し、文字ベースのインプット(参考書、公式ドキュメント、AWS Blogsなど)を重視して学習してきました。

また、試験を受け続けているうちに「試験の数週間前にこの状態(例: 問題演習で65\%程度)なら合格できる」といった感覚も掴めてきます。
この感覚が育つと、無理な詰め込みをせずに、淡々と積み上げられるようになりました。

\section{AWS認定とは・全冠とは}
本書で扱う「AWS認定」とは、Amazon Web Servicesが提供するクラウドの専門知識を証明する資格制度です。
基礎レベルの「Foundational」、中級の「Associate」、上級の「Professional」、そして特定技術に特化した「Specialty」と、役割や難易度に応じて複数の区分があります。

これら現在提供されているすべてのアクティブな認定資格を取得することを、界隈では敬意と達成感を込めて\textbf{「全冠(ぜんかん)」}と呼びます。\footnote{AWS Certification(公式): \url{https://aws.amazon.com/jp/certification/}}

本書は、\textbf{AWS Certified AI Practitioner(AIP-C01)}を含む、全試験の攻略に対応することを目指します。\footnote{AWS Certified AI Practitioner(公式): \url{https://aws.amazon.com/jp/certification/certified-ai-practitioner/}}
単一の資格を取るだけでも価値はありますが、「全冠」には独自の景色があります。インフラ、開発、セキュリティ、データ分析、そしてAI。これら全方位の知識がつながったとき、AWSという巨大なプラットフォームの「思想」が理解できるようになります。

\section{全冠のメリット}
全冠のメリットは、単に「バッジが増える」ことではありません。私の経験では、次の3つが大きいです。

\begin{itemize}
	\item \textbf{学習の地図が手に入る}: サービス同士のつながり(どこで何を使うか)が見えるようになります
	\item \textbf{説明の質が上がる}: 要件と制約から設計を選び、その理由を言語化しやすくなります
	\item \textbf{チャンスが増える場合がある}: 社内外で「頑張りを可視化」でき、声がかかるきっかけになります
\end{itemize}

\section{全冠は才能ではない}
多くの受験者は特別な才能を持っているわけではありません。適切な学習計画、反復、実践演習により誰でも達成可能です。本書では短期的な詰め込みではなく、持続的に学べる仕組みを重視します。

第1節の自己紹介で触れた通り、私は「非IT専攻」「激務」「実務経験浅め」という三重苦(?)のスタート地点にいました。それでも全冠を達成できた理由は、\textbf{「試験攻略をハックしたから」}に他なりません。

表紙にもある通り、AWS認定試験には攻略可能な\textbf{「メソッド」}が存在します。長文問題から正解を導き出す「四択テクニック」、明らかな誤答を削ぎ落とす「消去法」、そして難関とされる「複数回答」へのアプローチ。これらは、IQの高さやセンスに依存するものではなく、知っていれば誰でも使える技術です。

全冠は、才能の証明ではありません。「正しい努力を継続できる」という、エンジニアとしての基礎体力の証明です。そしてその「努力」は、本書のメソッドを使えば、最小限の労力で済むようになります。


\section{本書の対象読者}
本書は、以下のような方に向けて書いています。
\begin{itemize}
	\item 最近AWS認定を取り始め、クラウドの世界に興味を持ち始めた人
	\begin{itemize}
		\item CLFやSAAを取ってみて、「もっと深く知りたい」と思い始めた方に、ロードマップを提示します。
	\end{itemize}
	\item 「実務経験がないから」と上位資格を諦めかけている人
	\begin{itemize}
		\item 私の経歴が示す通り、実務経験が十分でなくても合格は狙えます。
	\end{itemize}
	\item 忙しくて勉強時間が取りにくいエンジニア
	\begin{itemize}
		\item 残業続きの私が実践した「隙間時間の活用」や「試験対策のショートカット術」を共有します。
	\end{itemize}
\end{itemize}

これから始まる各章では、私が実践の中で体系化した「合格への最短ルート」を余すところなくお伝えします。
それでは、第2章で具体的な「受かり方」の真髄に触れていきましょう。

\section{本書の読み方}
\begin{itemize}
	\item まずは章の冒頭(導入・できるようになること)を読み、到達点を確認してください。
	\item その後、手順やチェックリストを使って、学習を「問題が解ける形」に落とし込んでください。
	\item 手を動かせる箇所は、できる範囲で検証し、メモ(何を試し、何が分かったか)を残すのがおすすめです。
\end{itemize}

\section{まとめ}
\begin{tcolorbox}[colback=green!3, colframe=green!60!black]
\begin{itemize}
	\item 全冠は才能ではなく、戦略と継続の成果です
	\item 本書は「問題が解ける形」に落とし込むメソッドを提供します
	\item まずは読み方を掴み、次章から手順通りに進めていきます
\end{itemize}
\end{tcolorbox}