\chapter{はじめに}

\section{自己紹介}
「AWSの全冠(全認定取得)」と聞くと、皆さんはどのような人物を想像するでしょうか?
クラウドの黎明期から業界を牽引してきたベテランエンジニア、あるいは寝食を忘れて技術書を読み漁る一部の天才たち……。もしあなたがそう思っているなら、まずはその誤解を解くところから始めさせてください。
筆者である私は現在、社会人3年目の25歳です。大学時代は非情報系の学部に所属しており、プログラミングといえばJavaとPythonを少し独学で触った程度。ネットワークやサーバーといったインフラの知識は、入社するまで完全にゼロでした。
社会人1年目に配属されたインフラ部署で、わずか3ヶ月間だけAWSに触れる機会があり、そこで最初の資格「Cloud Practitioner (CLF)」を取得しました。その後はAIプロダクトの開発に携わり、ECSやEFSといった基本的なサービスを触る経験はしましたが、決して「AWSのスペシャリスト」としてバリバリ実務をこなしていたわけではありません。
さらに言えば、現在はAWSパートナー(APN)企業に所属していますが、担当業務はAWSとは無関係のシステム開発を行うSEです。つまり、「日々の業務でAWSスキルが自然と身につく環境」にはいませんでした。
そんな私が、平均月40時間以上の残業をこなしながら、約1ヶ月に1個のペースで試験を突破し、社会人3年目で全冠を達成できたのです。
私は、自分自身を「まだまだ未熟者」だと思っています。だからこそ断言できます。全冠達成に特別な才能は不要です。必要なのは、正しい戦略と、少しの工夫だけなのです。

\section{AWS認定とは・全冠とは}
本書で扱う「AWS認定」とは、Amazon Web Servicesが提供するクラウドの専門知識を証明する資格制度です。
基礎レベルの「Foundational」、中級の「Associate」、上級の「Professional」、そして特定技術に特化した「Specialty」と、役割や難易度に応じて複数の区分があります。これら現在提供されているすべてのアクティブな認定資格を取得することを、界隈では敬意と達成感を込めて**「全冠(ぜんかん)」**と呼びます。
本書は、最新の**「AWS Certified AI Practitioner (AIP-C01)」**1 を含む、全試験の攻略に対応しています。
単一の資格を取るだけでも価値はありますが、「全冠」には独自の景色があります。インフラ、開発、セキュリティ、データ分析、そしてAI。これら全方位の知識がつながったとき、AWSという巨大なプラットフォームの「思想」が理解できるようになります。

\section{全冠は才能ではない}
多くの受験者は特別な才能を持っているわけではありません。適切な学習計画、反復、実践演習により誰でも達成可能です。本書では短期的な詰め込みではなく、持続的に学べる仕組みを重視します。
第1節の自己紹介で触れた通り、私は「非IT専攻」「激務」「実務経験浅め」という三重苦(?)のスタート地点にいました。それでも全冠を達成できた理由は、**「試験攻略をハックしたから」**に他なりません。
表紙にもある通り、AWS認定試験には攻略可能な**「メソッド」が存在します 1。長文問題から正解を導き出す「四択テクニック」、明らかな誤答を削ぎ落とす「消去法」、そして難関とされる「複数回答」へのアプローチ 1。これらは、IQの高さやセンスに依存するものではなく、知っていれば誰でも使える技術**です。
全冠は、才能の証明ではありません。「正しい努力を継続できる」という、エンジニアとしての基礎体力の証明です。そしてその「努力」は、本書のメソッドを使えば、最小限の労力で済むようになります。


\section{本書の対象読者}
最近 AWS 認定に合格した、あるいは2つ目の資格を取得したばかりの初心者で、全冠に興味が湧いてきた人。
本書は、以下のような方に向けて書かれています。
最近AWS認定を取り始め、クラウドの世界に興味を持ち始めた人
CLFやSAAを取ってみて、「もっと深く知りたい」と思い始めたあなたに、最適なロードマップを提示します。
「実務経験がないから」と上位資格を諦めている人
私の経歴が証明するように、実務経験が伴わなくても合格は可能です。むしろ、資格学習で得た知識が、後の実務へのパスポートになります。
忙しくて勉強時間が取れないエンジニア
残業続きの私が実践した「隙間時間の活用法」や「試験対策のショートカット術」は、きっとあなたの武器になるはずです。
これから始まる各章では、私が実践の中で体系化した「合格への最短ルート」を余すところなくお伝えします。
それでは、第2章で具体的な「受かり方」の真髄に触れていきましょう。

\section{本書の読み方}
\begin{itemize}
	\item 本書は図解・フローを多用します。まず図を見て全体像をつかみ、その後詳細に入ってください。
	\item 各章末には「実習」「確認問題」「ワークシート」を用意します。手を動かして理解を深めましょう。
	\item 進捗管理用のチェックリストを巻末に用意しています。合格計画の記録に使ってください。
\end{itemize}

\section{この章で得られること}
\begin{itemize}
	\item 本書の目的と読み方が分かる
	\item 全冠に向けた大まかなロードマップを把握できる
	\item 実習の進め方と学習習慣の作り方が分かる
\end{itemize}